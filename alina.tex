\documentclass{article}
\usepackage{cmap} %копирование и поиск по файлам PDF
\usepackage[T2A]{fontenc} %подключение cyrillic шрифтов
\usepackage[russian]{babel} %корректное использование cyrillic символов
\usepackage[utf8]{inputenc} %кодировка
\usepackage{amssymb} %amsfonts +  дополнительные математические символов
\usepackage[dvipsnames]{xcolor} %дополнительные цвета
\pagestyle{plain} %заголовок пустой, колонтикул содержит номера страниц
\usepackage[unicode, colorlinks]{hyperref}  %оформление гиперссылок

%--------------------------------------
%Ненужные пакеты и команды:

%\usepackage{xcolor} -- не нужно, так как базовые цвета входят в пакет [dvipsnames]
%\newtheorem{Example}{Пример}[section] -- уже есть такая же команда, только с названием Ex
%\setcounter{MaxMatrixCols}{48} -- задание максимального числа столбцов в массиве

%\AtBeginDocument{
%  \hypersetup{
%    citecolor=Crimson,
 %  linkcolor = Crimson}}	-- зачем?
 
 %\usepackage{upgreek}  -- греческие буквы с надлежащим масштабированием в индексах  
  %\usepackage{amsthm} -- оформление теорем
  
 %\usepackage[shortlabels]{enumitem} -- краткий метод определения схемы нумерации 
 %\usepackage[tbtags]{amsmath} -- много дополнительных математических символов и команд
 %\usepackage{mathrsfs} -- использование Ralph Smith's Formal Script Font 
 %\usepackage[matrix,arrow,curve]{xy} -- очень интересные диаграммы со стрелочками
%\usepackage{graphicx} -- операции с графиками
%\usepackage{tikz} -- создание графических элементов
%\usetikzlibrary{calc, intersections} -- расчет точек пересечения элементов
%\usetikzlibrary {positioning} -- задание положения элементов друг относительно друга
%\usetikzlibrary{decorations.markings} -- линии/графики + преобразования

% Рисование и отображение
%\usepackage{pgfplots,caption} -- рисование корректных графиков и картинок
%\pgfplotsset{compat=1.9} -- загружаемая версия
%\usepackage{placeins}  -- корректное отображение чисел с запятой
%\usepackage{resizegather}  -- попытка уместить большие уравнения в строку 

%красивые буквы
%\newcommand{\E}{\mathbb{E}}
%\newcommand{\LL}{\mathcal{L}}
%\newcommand{\QQ}{\mathcal{Q}}
%\newcommand{\LIND}{\mathcal{L}_{\textup{ind}}}

%зачем, если есть команда Anyletter
%\newcommand{\p}[2]{p^{#1}_{#2}}
%\newcommand{\h}[2]{h^{#1}_{#2}}
%\newcommand{\q}[2]{q^{#1}_{#2}}
%\newcommand{\g}[2]{g^{#1}_{#2}}

%скобочки
%\newcommand{\absbrackets}[1]{\left|#1 \right|}
% ссылки в скобочках
%\let\oldref\ref
%\renewcommand{\ref}[1]{(\oldref{#1})}

%определения
%\newtheorem{Def}{Определение}[section]
%\newtheorem{Ex}{Пример}[section]
%\newtheorem{St}{Утверждение}[section]
%\newtheorem{Pred}{Предложение}[section]
%\newtheorem{Zam}{Замечание}[section]

% пользовательские символы
%\newcommand{\norm}[1]{\left\lVert#1\right\rVert}    %норма
%\newcommand{\vect}[1]{\boldsymbol{#1}}					%вектор
%\def\defeq{\mathrel{\stackrel{\rm def}=}}   				%равно по определению
%--------------------------------------

% Геометрия текста
\textheight=24cm            	% высота текста
\textwidth=16cm				% ширина текста
\oddsidemargin=0pt          % отступ от левого края
\topmargin=-1.5cm           % отступ от верхнего края
\parindent=24pt					% абзацный отступ
\parskip=4pt						% интервал между абзацами
\def\baselinestretch{1.2}		%интервал между строками
	
%--------------------------------------

% Определения
\newtheorem{Lem}{Лемма}[section]
\newtheorem{Th}{Теорема}[section]

% Пользовательские символы
\newcommand{\anyletter}[3]{#1^{#2}_{#3}}

%красивые буквы
\newcommand{\PP}{\mathsf{P}} 
\newcommand{\LEMB}{\mathcal{L}_{\textup{emb}}}

%дурость, но оставлю
\newcommand{\question}[1]{\textcolor{PineGreen}{\textit{(!!! #1  !!!)}}}

% красивые скобочки
\let\originalleft\left
\let\originalright\right
\renewcommand{\left}{\mathopen{}\mathclose\bgroup\originalleft}
\renewcommand{\right}{\aftergroup\egroup\originalright}
\newcommand{\brackets}[1]{\left(#1 \right)}
\newcommand{\PCOND}[2]{\PP\brackets{#1 \: | \: #2}}


\newenvironment{Proof}
{\par\noindent{\bf Доказательство}}
{\hfill$\scriptstyle\blacksquare$} 
%--------------------------------------
    
\begin{document}
\section{Вероятности достижения}
\noindent В случае $M \question{смотрю} \leq 0$ верно, что $\PP\brackets{\tau_{x_0} < \infty} = 1$. Поэтому справедливо следующее равенство
\begin{equation}\label{eq:h_xi_2_only}
\anyletter{h}{x_0, i_0}{0,i} = \PP \brackets{\xi_{\tau_{x_0}}^{2} = i \: | \: \xi_0 = (x_0, i_0)}.
\end{equation}

\noindent Определим \textit{вложенную марковскую} цепь $\LEMB$ как конечную марковскую цепь на пространстве состояний $\{0,1, \dots, N\}$ по моментам времени $\tau_k$.  Пусть $\xi_{\tau_k}^{2}$ -- состояние цепи $\LEMB$ в момент времени $\tau_k$. Переходные вероятности запишем как:
\[
p(i \rightarrow j) = \anyletter{v}{i}{j}= \PP \brackets{\xi_{\tau_{k + 1}}^{2} = j \: | \: \xi_{\tau_k}^{2} = i}.
\]

\noindent Можно доказать, что вероятности достижения $\anyletter{h}{x_0, i_0}{0,i}$ могут быть выражены через переходные вероятности $\anyletter{v}{i}{j}$ следующим образом.
\begin{Lem}\label{lem:h_and_v_connection}
Пусть $M \leq 0$. Тогда
\[
\anyletter{h}{x_0, i_0}{0,i} = \sum_{i_1, i_2, \dots, i_{x_0 - 1}} \anyletter{v}{i_0}{i_1} \anyletter{v}{i_1}{i_2} \dots \anyletter{v}{i_{x_0-1}}{i} = \anyletter{v}{i_0}{i}(x_0),
\]
где $\anyletter{v}{i_0}{i}(k)$ -- переходная вероятность за $k$ шагов марковской цепи $\LEMB$.
\end{Lem}
\begin{Proof}
Положим $\anyletter{\hat{v}}{i}{j}(0) = 0$ и
\[
\anyletter{\hat{v}}{i}{j}(k) = \PP\brackets{\tau_1 = k, \xi_{\tau_1}^{2} = j \: | \: \xi_{0}^{2} = i} = \sum \anyletter{p}{x, i}{x_1, i_1}  \anyletter{p}{x_1, i_1}{x_2, i_2} \dots \ \anyletter{p}{x_{k - 1}, i_{k - 1}}{x_k, i_k}, 
\]
где сумма берётся по всем путям длины $k$, что $x_s \geq x, \: s = 1, \dots, k - 2, \: x_{k-1} = x, \: x_k = x - 1,  i_k = j$.  В силу условия однородности определение вероятностей $\anyletter{\hat{v}}{i}{j}(k)$ корректно, поскольку правая часть последнего равенства не зависит от $x$.

\noindent Тогда вероятности перехода $\anyletter{v}{i}{j}$ вложенной цепи $\LEMB$ можно выразить через вероятности $\anyletter{\hat{v}}{i}{j}(k)$ следующим образом:
\[
\anyletter{v}{i}{j} = \sum_{k = 1}^{\infty} \anyletter{\hat{v}}{i}{j}(k).
\]

\noindent Тогда вероятности $\anyletter{h}{x_0, i_0}{0,i}$ можно записать как
\[
\anyletter{h}{x_0, i_0}{0,i} = \sum_{n \geq x_0}\PP \brackets{\tau_{x_0} = n,  \xi_{\tau_{x_0}}^{2} = i \: | \:  \xi_{0} = (x_0, i_0)} = \sum_{\mathcal{A}} \sum_{n \geq x_0} \sum_{\mathcal{B}} \anyletter{\hat{v}}{i_0}{i_1}(k_1) \anyletter{\hat{v}}{i_1}{i_2}(k_2)\dots \anyletter{\hat{v}}{i_{x_0 -1}}{i}(k_{x_0}),
\]
где первая сумма берётся по множеству $\mathcal{A}$ всех наборов $(i_1, i_2, \dots, i_{x_0 - 1})$, а третья сумма -- по множеству $\mathcal{B}$ всех $k_l :  k_1 + \dots + k_{x_0} = n$
Тогда 
\[
\sum_{n \geq x_0} \sum_{\mathcal{B}} \anyletter{\hat{v}}{i_0}{i_1}(k_1) \anyletter{\hat{v}}{i_1}{i_2}(k_2)\dots \anyletter{\hat{v}}{i_{x_0 -1}}{i}(k_{x_0}) = \sum_{k_1 = 1}^{\infty}\sum_{k_2 = 1}^{\infty} \dots \sum_{k_{x_0} = 1}^{\infty} \anyletter{\hat{v}}{i_0}{i_1}(k_1)\anyletter{\hat{v}}{i_1}{i_2}(k_2)\dots \anyletter{\hat{v}}{i_{x_0 -1}}{i}(k_{x_0}) = \anyletter{v}{i_0}{i_1} \anyletter{v}{i_1}{i_2}\dots \anyletter{v}{i_{x_0 - 1}}{i}.
\]
Следовательно,
\[
\anyletter{h}{x_0, i_0}{0,i} = \sum_{\mathcal{A}} \anyletter{v}{i_0}{i_1} \anyletter{v}{i_1}{i_2}\dots \anyletter{v}{i_{x_0 - 1}}{i} = \anyletter{v}{i_0}{i}(x_0).
\]

\end{Proof}
\end{document}
