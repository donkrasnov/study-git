\documentclass{article}
\usepackage{cmap} %копирование и поиск по файлам PDF
\usepackage[T2A]{fontenc} %подключение cyrillic шрифтов
\usepackage[russian]{babel} %корректное использование cyrillic символов
\usepackage[utf8]{inputenc} %кодировка
\usepackage{amssymb} %amsfonts +  дополнительные математические символов
\usepackage{amsmath}
\usepackage{amsfonts}
\pagestyle{plain} %заголовок пустой, колонтикул содержит номера страниц

%красивые буквы
\newcommand{\E}{\mathbb{E}}
\newcommand{\RR}{\mathcal{R}}
\newcommand{\PP}{\mathsf{P}} 

%определения
\newtheorem{Def}{Определение}[section]
\newtheorem{Ex}{Пример}[section]
\newtheorem{Th}{Теорема}[section]
\newtheorem{Pred}{Предложение}[section]
\newtheorem{Zam}{Замечание}[section]

% Геометрия текста
\textheight=24cm            	% высота текста
\textwidth=16cm				% ширина текста
\oddsidemargin=0pt          % отступ от левого края
\topmargin=-1.5cm           % отступ от верхнего края
\parindent=24pt					% абзацный отступ
\parskip=4pt						% интервал между абзацами
\def\baselinestretch{1.2}		%интервал между строками

% Пользовательские символы
\newcommand{\anyletter}[3]{#1^{#2}_{#3}}

% красивые скобочки
\let\originalleft\left
\let\originalright\right
\renewcommand{\left}{\mathopen{}\mathclose\bgroup\originalleft}
\renewcommand{\right}{\aftergroup\egroup\originalright}
\newcommand{\brackets}[1]{\left(#1 \right)}
\newcommand{\PCOND}[2]{\PP\brackets{#1 \: | \: #2}}

\begin{document}

\section{Характеристические функции.}
$\zeta$ - случайная величина, но со значениями в $\mathbb {C}$.
$$\zeta = \xi + i\eta$$
$$\E\zeta = \E\xi + i\E\eta$$
$\mathbb {E}\zeta$ определено $\Longleftrightarrow$ $\mathbb {E}\xi + \mathbb {E}\eta$ определены.

\Def Пусть есть 2 комплекснозначные случайные величины:
$$\zeta_1 = \xi_1 + i\eta_1$$
$$\zeta_2 = \xi_2 + i\eta_2$$
Они независимы $\Longleftrightarrow$ независимы пары ($\xi_1, \xi_2$) и ($\eta_1, \eta_2$).

\Def Пусть $ F = F(x)$ - $n$-мерная функция распределения в ($\mathbb {R}^n$,$\mathcal B(\mathbb {R}^n)$), $x = (x_1, x_2, ..., x_n)$. \newline
Характеристической функцией 
($n$-мерной функцией распределения) называется функция:
$$\varphi(t) = \int_{\mathbb {R}^n} e^{i(t,x)}d{F(x)}, t \in \mathbb {R}^n$$

\Def  Пусть $\xi = (\xi_1, \xi_2, ..., \xi_n)$ - случайный вектор на $(\Omega, \mathcal{F}, \mathbb{P})$ со значениями в $\mathbb {R}^n$. Характеристической функцией случайного вектора $\xi$ называется функция: 
$$\varphi_{\xi}(t)  = \int_{\mathbb {R}^n} e^{i(t,x)}d{F_{\xi}(t)}, t \in \mathbb {R}^n$$
$F_{\xi} = F_{\xi}(t)$ -- функция распределения вектора $\xi = (\xi_1, \xi_2, ..., \xi_n)$, $x = (x_1, x_2, ..., x_n)$.

\noindent Пусть f(x) - плотность функции распределения F(x):
$$\varphi(t)  = \int_{\mathbb {R}} e^{i(t,x)}f(x)d{x}$$
$$\mathbb {E}g(\xi)  = \int_{\mathbb {R}^n} g(x)f_{\xi}(x)d{x}$$

\Zam {Начальные наблюдения о характеристических функциях}
\begin{enumerate}
\item $\eta = a\xi + b$ \newline
$\varphi_{\eta}(t) = \mathbb {E} e^{it\eta} = \mathbb {E} e^{it(a\xi + b)} =  \mathbb {E} e^{ita\xi} e^{itb} = e^{itb}\mathbb {E}e^{ita\xi}$
$$\varphi_{\eta}(t) = e^{itb}\varphi_{\xi}(at)$$
\item $\xi_1, \xi_2, ..., \xi_n$ - независимые случайные величины, 
$S_n = \xi_1+ \xi_2+ ... + \xi_n$ \newline
${\varphi_S}_n(t) = \mathbb{E}e^{it(\xi_1+ \xi_2+ ... + \xi_n)} = \mathbb{E}e^{it\xi_1}e^{it\xi_2}...e^{it\xi_n} =  \mathbb{E}e^{it\xi_1}\mathbb{E}e^{it\xi_2}...\mathbb{E}e^{it\xi_n} = \prod\limits_{j =1}^n{\varphi_{\xi}}_j(t)$
$${\varphi_S}_n(t) = \prod\limits_{j =1}^n{\varphi_{\xi}}_j(t)$$
\end{enumerate}
\Ex
\begin{enumerate}
\item $\xi$ ~ $Bern(p), 0<p<1$
$$\varphi_{\xi}(t) = \mathbb{E}e^{it\xi} = e^{it*0}*\mathbb {P}{\xi = 0} + e^{it*1}*\mathbb {P}{\xi=1} = (1-p)+ e^{it}p = q + e^{it}p$$
\item Пусть $\xi_1, \xi_2, ..., \xi_n$ - независимые случайные величины с одинаковым распределением Бернулли. Пусть $T_n = \frac{S_n - np}{ \sqrt{npq}}$
$${\varphi_T}_n(t) = \varphi_{\frac{S_n - np}{\sqrt{npq}}}(t) = \mathbb {E}e^{it\frac{S_n - np}{ \sqrt{npq}}} =\mathbb {E}e^{it \frac{S_n}{\sqrt{npq}}}e^{-it\sqrt{\frac{np}{q}}} = e^{-it\sqrt{\frac{np}{q}}} (q+e^{\frac{it}{\sqrt{npq}}}p)^n = (qe^{-it\sqrt{\frac{p}{nq}}} + pe^{it\sqrt{\frac{q}{np}}})^n$$
$${\varphi_T}_n(t)\underset{n \to \infty}{\longrightarrow} e^{-\frac{t^2}{2}}$$
\end{enumerate}

\textbf{Свойства характеристических функций}
\Th
Пусть $\xi = \xi(\omega)$ - случайная величина, $F = F(x)$ - функция распределения $\xi$.
$\varphi_{\xi}(t) = \mathbb{E}e^{it\xi}$  - характеристическая функция. Тогда:
\begin{enumerate}
\item $|\varphi_{\xi}(t)| \leqslant \varphi_{\xi}(o) = 1$
\item $\varphi_{\xi}(t)$ равномерно непрерывна по t
\item $\varphi_{\xi}(t) = \overline{\varphi_{\xi}(-t)} $
\item Если $\exists n: n \geqslant 1, E|\xi|^n < \infty$, то $\forall r \leqslant n \exists \varphi^{(r)}_{\xi}(t)$ и $\varphi^{(r)}_{\xi}(t) = \int_{\mathbb {R}} (ix)^re^{itx}d{F(x)}$
$$ E\xi^r = \frac{\varphi^{(r)}(o)}{i^r}$$
$\varphi^{(r)}_{\xi}(t) = \sum\limits_{к=0}^n \frac{(it)^r}{r!}E\xi^r + \frac{(it)^n}{n!}\epsilon_n(t)$, где $\epsilon_n(t) \underset{t \to 0}{\longrightarrow} 0$
\item Если $\exists$ конечная $\varphi_{\xi}^(2n)(o)$, то $E\xi^{(2n)}< \infty$  
\end{enumerate}

\Th{(об определении функции распределения по характеристическим функциям).}
Пусть F и G - функции распределения. При этом если функции F и G совпадают, т.е. $\forall  t \in \mathbb {R}  \int_{\mathbb {R}} e^{itx}d{F(x)} = \int_{\mathbb {R}} e^{itx}d{G(x)}$. Тогда $F(x) = G(x)$.

\Th{(обращение).}
Пусть $F = F_\xi$ - функция распределения, $\varphi_{\xi}(t) =\int_{\mathbb {R}} e^{itx}d{F(x)} $ - характеристическая функция. Тогда: $\forall {a,b}: a<b, F \in C(a,b)$
$F(b) - F(a) = \lim\limits_{c\to\infty} \frac{1}{2\pi} \int\limits_{-c}^c \frac{e^{-ita}-e^{-itb}}{it} \varphi(t)dt$
Если $\int_{\mathbb {R}} |\varphi(t)|dt <  \infty$ , то функция распределения F(x) имеет плотность  f(x).
$$F(x) = \int_{-\infty}^x f(y)dy$$
$$f(x) = \frac{1}{2\pi}\int_{\mathbb {R}}e^{-itx}\varphi(t)dt$$
\end{document}

